\documentclass{skvitae}
\usepackage{polyglossia}
\setmainlanguage{swedish}
\usepackage{csquotes}
\usepackage{ragged2e}

\SetupContactInfo{
	name		= {Simon Sigurdhsson},
	title		= {Mjukvaruingenjör},
	company = {Evidente~AB},
	street  = {Björkrisvägen 43},
	zip 		= {42243},
	city 		= {Hisings~Backa},
	%country = {},
	email 	= {Sigurdhsson@gmail.com},
	web 		= {http://sigurdhsson.org},
	%phone   = {},
	mobile  = {0703293054},
	github  = {urdh},
}

\begin{document}
	\RaggedRight%
	\maketitle

	%\section{Sammanfattning}
	%\item 50-60 ord

	\section{Utbildning}
	\EduItem{
		from = 2008, to = 2014,
		school = {Chalmers Tekniska Högskola},
		level = {Civilingenjör},
		field = {Teknisk Matematik},
		extra = {
			\Thesis{type = msc, language = english,
							title = {Solving max-sum problems with the in-the-middle heuristic}},
			%{Masterprogram: \emph{Engineering Mathematics \and\ Computational Science}},
			%\Thesis{type = bsc, language = swedish,
			%				title = {Statistisk bildanalys av handgester för människa-dator-interaktion}}
		}
	}

	\section{Uppdrag}
  \WorkItem{from = 2016, to = \Current, city = {Göteborg},
            title = {Mjukvaruingenjör}, org = {Evidente~AB}}
	\WorkItem{from = 2014, to = 2016, city = {Göteborg},
						title = {Systemutvecklare, \emph{O\and M Development}}, org = {Ericsson~AB}}
	\WorkItem{from = 2013, to = 2013, city = {Göteborg},
						title = {Sommarjobb (utveckare inom \emph{Verification efficiency})}, org = {Volvo Personvagnar~AB}}
	%\WorkItem{from = 2006, to = 2008, city = {Trelleborg},
	%					title = {Reklamutdelare}, org = {SDR Svensk Direktreklam}}

	\subsection{Ideella uppdrag}
	%\WorkItem{from = 2012, to = 2013, city = {Göteborg},
	%            title = {Redaktör}, org = {Fysikteknologsektionens informationsblad}}
	\WorkItem{from = 2012, to = 2013, city = {Göteborg}, glue-word = {i},
						title = {Ekonomiskt ansvarig}, org = {Chalmers Studentkårs Film- och Fotocommitté}}
	\WorkItem{from = 2011, to = 2014, city = {Göteborg}, glue-word = {i},
						title = {Fotograf}, org = {Chalmers Studentkårs Film- och Fotocommitté}}
	\WorkItem{from = 2010, to = 2011, city = {Göteborg}, glue-word = {i},
						title = {Sekreterare}, org = {Fysikteknologsektionens styrelse}}
	\WorkItem{from = 2009, to = 2010, city = {Göteborg}, glue-word = {i},
						title = {Ledamot}, org = {Fysikteknologsektionens sexmästeri}}
	\WorkItem{from = 2009, to = 2011, city = {Göteborg},
						title = {Redaktör}, org = {Fysikteknologsektionens informationsblad}}

	\section{Kompetenser}
	\subsection{Språk}
	\begin{itemize}
		\item Svenska (modersmål).
		\item Engelska (flytande i tal och skrift).
	\end{itemize}

	\subsection{IT-kompetenser}
	\begin{itemize}
		\item Goda kunskaper: C, C++, Git, Google Test, \LaTeX, MATLAB, Python, R, Ruby, TTCN-3.
		\item Grundläggande kunskaper: CANalyzer, C\#.NET, Erlang, Fortran, Java, Mathematica, Subversion.
	\end{itemize}

	\subsection{Övrigt}
	\begin{itemize}
		\item Innehar B-körkort sedan 2008.
		\item Referenser lämnas på begäran.
	\end{itemize}
\end{document}
